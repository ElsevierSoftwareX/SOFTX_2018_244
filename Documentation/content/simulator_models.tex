\chapter{Simulator Models}

\section{Basic Components}

\subsection{Inductor}

\begin{equation}
        \frac{di(t)}{dt} = \frac{1}{L} \cdot v(t) - j \omega \cdot i(t)
\end{equation}

Apply trapezoidal rule:

\begin{equation}
        i(t) = i(t - \Delta t) + \frac{\Delta t}{2} \left[ \frac{1}{L} \cdot v(t) - j \omega \cdot i(t) + \frac{1}{L} \cdot v(t - \Delta t) - j \omega \cdot i(t - \Delta t) \right]
\end{equation}

\begin{align}
        a &= \frac{\Delta t}{2L} \\
        b &= \frac{\Delta t \omega}{2}
\end{align}

\begin{align}
        i(t) &= i(t - \Delta t) + a \cdot v(t) - j b \cdot i(t) + a \cdot v(t - \Delta t) - j b \cdot i(t - \Delta t) \\
        i(t) &= \frac{1-b^2-j2b}{1+b^2} \cdot i(t - \Delta t) + \frac{a-jab}{1+b^2} \left[ v(t) - v(t - \Delta t) \right]
\end{align}

\subsection{Capacitor}

\begin{equation}
        \frac{dv(t)}{dt} = \frac{1}{C} \cdot i(t) - j \omega \cdot v(t)
\end{equation}

Apply trapezoidal rule:

\begin{equation}
        v(t) = v(t - \Delta t) + \frac{\Delta t}{2} \left[ \frac{1}{C} \cdot i(t) - j \omega \cdot v(t) + \frac{1}{C} \cdot i(t - \Delta t) - j \omega \cdot v(t - \Delta t) \right]
\end{equation}

\begin{align}
        a &= \frac{\Delta t}{2C} \\
        b &= \frac{\Delta t \omega}{2}
\end{align}

\begin{align}
        v(t) &= v(t - \Delta t) + a \cdot i(t) - j b \cdot v(t) + a \cdot i(t - \Delta t) - j b \cdot v(t - \Delta t) \\
        i(t) &= -i(t- \Delta t) + \frac{1+jb}{a} \cdot v(t) + \frac{-1+jb}{a} \cdot v(t - \Delta t)
\end{align}

\section{Synchronous Machine}

The model is according to \cite{wang2010methods} and \cite{kundur1994power}. 

\subsubsection{Prerequisites}
Park's transformation is commonly used to achieve a model with static parameters:
%
\begin{equation}
\mathbf{K_s} = \frac{2}{3}
 \begin{bmatrix} 
  \cos \theta & \cos(\theta-\frac{2\pi}{3}) & \cos(\theta+\frac{2\pi}{3}) \\
  \sin \theta & \sin(\theta-\frac{2\pi}{3}) & \sin(\theta+\frac{2\pi}{3}) \\
  \frac{1}{2} & \frac{1}{2} & \frac{1}{2}
 \end{bmatrix}
\end{equation}
%
Note that the scaling factor $\frac{2}{3}$ is not always used. 

\subsubsection{Model}

The mechanical equations are:
%
\begin{align}
\frac{d\theta_r}{dt} &= \omega_r \\
\frac{d\omega_r}{dt} &= \frac{P}{2J} (T_e-T_m)
\end{align}
%
where $\theta_r$ is the rotor position, $\omega_r$ is the angular electrical speed, $P$ is the number of poles, $J$ is the moment of inertia, $T_m$ and $T_e$ are the mechanical and electrical torque, respectively. Motor convention is used for all models. 

The electrical model in the phase domain is described by the following equations:
%
\begin{align}
  \mathbf{v}_{abcs} &= \mathbf{R}_s \mathbf{i}_{abcs} + \frac{d}{dt} \boldsymbol{\lambda}_{abcs} \\
  \mathbf{v}_{qdr} &= \mathbf{R}_r \mathbf{i}_{qdr} + \frac{d}{dt}  \boldsymbol{\lambda}_{qdr}
\end{align}
%
where
%
\begin{align}
  \mathbf{v}_{abcs} &= 
  \begin{bmatrix}
    v_{as} & v_{bs} & v_{cs}
  \end{bmatrix}^T \\
  %  
  \mathbf{v}_{qdr} &= 
  \begin{bmatrix}
    v_{kq1} & v_{kq2} & v_{fd} & v_{kd} 
  \end{bmatrix}^T \\
  %
  \mathbf{i}_{abcs} &= 
  \begin{bmatrix}
    i_{as} & i_{bs} & i_{cs}
  \end{bmatrix}^T \\
  %  
  \mathbf{i}_{qdr} &= 
  \begin{bmatrix}
    i_{kq1} & i_{kq2} & i_{fd} & i_{kd} 
  \end{bmatrix}^T \\
  %
  \boldsymbol{\lambda}_{abcs} &= 
  \begin{bmatrix}
    \lambda_{as} & \lambda_{bs} & \lambda_{cs}
  \end{bmatrix}^T \\
  %  
  \boldsymbol{\lambda}_{qdr} &= 
  \begin{bmatrix}
    \lambda_{kq1} & \lambda_{kq2} & \lambda_{fd} & \lambda_{kd} 
  \end{bmatrix}^T \\
  %
  \mathbf{R}_s &= diag
  \begin{bmatrix}
    r_s & r_s & r_s 
  \end{bmatrix} \\
  %
  \mathbf{R}_r &= diag
  \begin{bmatrix}
    r_{kq1} & r_{kq2} & r_{fd} & r_{kd}
  \end{bmatrix}
\end{align}
%
The flux linkages are:
%
\begin{align}
  \boldsymbol{\lambda}_{abcs} &= \mathbf{L}_{ss} \mathbf{i}_{abcs} + \mathbf{L}_{sr} \mathbf{i}_{qdr}  \\
  \boldsymbol{\lambda}_{qdr} &= (\mathbf{L}_{sr})^T \mathbf{i}_{abcs} + \mathbf{L}_{rr} \mathbf{i}_{qdr}
\end{align}
%
where the inductances are time variant variables as defined in \cite{krause2002sudhoff}.
TODO: electromagnetic torque

\subsubsection{Equations in Rotor Reference Frame}

This section depicts the synchronous generator equations in terms of machine variables referred to the stator windings which is indicated by the prime symbol. Due to the transform of the stator variables to the rotor reference frame, the stator equations change, whereas the rotor equations remain the same.
%
\begin{align}
  \mathbf{v}_{dq0s} &= \mathbf{R}_{qds} \mathbf{i}_{qd0s} + \frac{d}{dt} \boldsymbol{\lambda}_{qd0s} + \omega_r \boldsymbol{\lambda}_{dqs} \\
  \mathbf{v}_{qdr}' &= \mathbf{R}_r' \mathbf{i}_{qdr}' + \frac{d}{dt}  \boldsymbol{\lambda}_{qdr}'
\end{align}
%
where
%
\begin{align}
  (\boldsymbol{\lambda}_{dqs})^T &= 
  \begin{bmatrix}
    \lambda_{ds} & -\lambda_{qs} & 0
  \end{bmatrix}
\end{align}
%
The flux linkages are:
%
\begin{align}
  \boldsymbol{\lambda}_{qd0s} &= \mathbf{L}_{qdss} \mathbf{i}_{qd0s} + \mathbf{L}_{qdsr} \mathbf{i}_{qdr}'  \\
  \boldsymbol{\lambda}_{qdr} &= \mathbf{L}_{qdfs} \mathbf{i}_{qd0s} + \mathbf{L}_{qdff} \mathbf{i}_{qdr}'
\end{align}
%
where
%
\begin{align}
  \mathbf{L}_{qdss} &= 
  \begin{bmatrix}
    L_{q} & 0 & 0 \\
    0 & L_{q} & 0 \\
    0 & 0 & L_{ls}
  \end{bmatrix} \\
  %  
  \mathbf{L}_{qdsr} &= 
  \begin{bmatrix}
    L_{mq} & L_{mq} & 0 & 0 \\
    0 & 0 & L_{md} & L_{md} \\
    0 & 0 & 0 & 0
  \end{bmatrix} \\
  %
  \mathbf{L}_{qdfs} &=
  \begin{bmatrix}
    L_{mq} & 0 & 0 \\
    L_{mq} & 0 & 0 \\
    0 & L_{md} & 0 \\
    0 & L_{md} & 0
  \end{bmatrix} \\
  %
  \mathbf{L}_{qdff} &=
  \begin{bmatrix}
    L_{kq1} & L_{mq} & 0 & 0 \\
    L_{mq} & L_{kq2} & 0 & 0 \\
    0 & 0 & L_{fd} & L_{md} \\
    0 & 0 & L_{md} & L_{kd}
  \end{bmatrix} \\
  %
\end{align}
%
with 
%
\begin{align}
  L_{q} &= L_{ls} + L_{mq} \\
  L_{d} &= L_{ls} + L_{md} \\
  L_{kq1} &= L_{lkq1} + L_{mq} \\
  L_{kq2} &= L_{lkq2} + L_{mq} \\
  L_{fd} &= L_{lfd} + L_{md} \\
  L_{kd} &= L_{lkd} + L_{md}
\end{align}
\subsubsection{State Space Model}
%
\begin{equation}
  \frac{d}{dt}
  \begin{pmatrix}
    \mathbf{i}_{qd0s} \\
    \mathbf{i}_{qdr}
  \end{pmatrix}
  = \mathbf{L}^{-1} \left[
  \begin{pmatrix}
    \mathbf{v}_{qd0s} \\
    \mathbf{v}_{qdr}
  \end{pmatrix}
  - \mathbf{R}
  \begin{pmatrix}
    \mathbf{i}_{qd0s} \\
    \mathbf{i}_{qdr}
  \end{pmatrix}
  - \omega_r
  \begin{pmatrix}
    \boldsymbol{\lambda}_{dq} \\
    0
  \end{pmatrix}
  \right]
\end{equation}
%
